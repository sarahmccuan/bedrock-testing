\section*{7.33}

Οἰόμενος δ' ἡμᾶς διὰ τὸ γνῶναι καὶ ἰδεῖν τὸν θεὸν πρεσβεύειν\footnoteA{πρεσβεύω, advocate, argue for, inf. pres. act.} τὰ περὶ τῆς ἀναστάσεως συνείρει\footnoteA{συνείρω, string together, connect, 3rd sg. pres. act. ind.} ἑαυτῷ ἃ βούλεται καὶ τοιαῦτά φησιν· ὅταν δὴ πάντοθεν ἐξείργωνται\footnoteA{ἐξείργω, exclude, shut out, 3rd pl. perf. pass. subj.} καὶ διελέγχωνται\footnoteA{διελέγχω, refute thoroughly, cross-examine, 3rd pl. pres. pass. subj.}, πάλιν ὥσπερ οὐδὲν ἀκηκοότες ἐπανίασιν\footnoteA{ἐπάνειμι, return, go back to, 3rd pl. pres. act. ind.} ἐπὶ τὸ αὐτὸ ἐρώτημα· πῶς οὖν γνῶμεν καὶ ἴδωμεν τὸν θεόν; καὶ πῶς ἴωμεν πρὸς αὐτόν; ἴστω οὖν ὁ βουλόμενος ὅτι, κἂν εἰς ἄλλα δεώμεθα\footnoteA{δέομαι, need, require, 1st pl. pres. mid./pass. subj.} σώματος τῷ ἐν τόπῳ σωματικῷ τυγχάνειν\footnoteA{τυγχάνω, happen to be, be present, inf. pres. act.}, καὶ τοιούτου, ὁποία ἐστὶν ἡ φύσις τοῦ σωματικοῦ τόπου, καὶ δεόμενοι τοῦ σώματος ἐπενδυώμεθα\footnoteA{ἐπενδύω, put on over, clothe oneself with, 1st pl. pres. mid. subj.} τῷ σκήνει\footnoteA{σκῆνος, tent, tabernacle, dat. sg. neut.} τὰ προειρημένα, ἀλλ' εἰς γνῶσίν γε θεοῦ σώματος οὐδαμῶς χρῄζομεν\footnoteA{χρῄζω, need, have need of, 1st pl. pres. act. ind.}. τὸ γὰρ γινῶσκον θεὸν οὐκ ὀφθαλμός ἐστι σώματος ἀλλὰ νοῦς, ὁρῶν τὸ «κατ' εἰκόνα» τοῦ κτίσαντος καὶ τὸ δυνάμενον γινώσκειν θεὸν προνοίᾳ θεοῦ ἀνειληφώς\footnoteA{ἀναλαμβάνω, take up, receive, perf. act. part. nom. sg. masc.}. καὶ τὸ ὁρῶν δὲ θεὸν καθαρά ἐστι καρδία, ἀφ' ἧς οὐκέτι «ἐξέρχονται διαλογισμοὶ πονηροὶ,» οὐ «φόνοι,» οὐ «μοιχεῖαι,» οὐ «πορνεῖαι,» οὐ «κλοπαὶ,» οὐ «ψευδομαρτυρίαι\footnoteA{ψευδομαρτυρία, false witness, perjury, nom. pl. fem.},» οὐ «βλασφημίαι\footnoteA{βλασφημία, blasphemy, slander, nom. pl. fem.},» οὐκ «ὀφθαλμὸς πονηρὸς» οὐδ' ἄλλο τι τῶν ἀτόπων\footnoteA{ἄτοπος, out of place, improper, gen. pl. neut.}· δι' ἃ λέγεται· «μακάριοι οἱ καθαροὶ τῇ καρδίᾳ, ὅτι αὐτοὶ τὸν θεὸν ὄψονται.» ἐπεὶ δ' οὐκ αὐτάρκης\footnoteA{αὐτάρκης, self-sufficient, adequate, nom. sg. fem.} ἡ ἡμετέρα προαίρεσις\footnoteA{προαίρεσις, choice, moral purpose, nom. sg. fem.} πρὸς τὸ πάντῃ «καθαρὰν» ἔχειν τὴν «καρδίαν,» ἀλλὰ θεοῦ ἡμῖν δεῖ, κτίζοντος αὐτὴν τοιαύτην, διὰ τοῦτο λέγεται ὑπὸ τοῦ ἐπιστημόνως\footnoteA{ἐπιστημόνως, skillfully, knowledgeably, adverb} εὐχομένου· «καρδίαν καθαρὰν κτίσον ἐν ἐμοὶ ὁ θεός.»