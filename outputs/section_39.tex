\section*{7.39}

Ἴδωμεν δὲ καὶ ἐπὶ τίνα ἡμᾶς καλεῖ, ἵν' αὐτοῦ ἀκούσωμεν, τίνι τρόπῳ γνωσόμεθα τὸν θεόν· ἐφ' οἷς οἴεται μηδένα Χριστιανῶν ἐπαΐειν\footnoteA{ἐπαΐω, understand, comprehend, pres. act. inf.} δύνασθαι τῶν ὑπ' αὐτοῦ λεγομένων, φησὶ γάρ· ὅμως δ' οὖν ἀκουσάτωσαν, εἴ τι καὶ ἐπαΐειν δύνανται. τίνων οὖν ἡμᾶς ἀκούειν ὑπ' αὐτοῦ λεγομένων βούλεται, κατανοητέον\footnoteA{κατανοέω, observe carefully, consider, verbal adj. neut. sg.}, ὁ φιλόσοφος. δέον διδάσκειν ἡμᾶς, ὁ δὲ διαλοιδορεῖται\footnoteA{διαλοιδορέω, revile, abuse, 3rd sg. pres. mid. ind.}· καὶ δέον εὔνοιαν ἑαυτοῦ δεῖξαι ἐν τῷ προοιμίῳ\footnoteA{προοίμιον, preface, introduction, dat. sg. neut.} τῶν λόγων τὴν πρὸς τοὺς ἀκούοντας, ὁ δέ φησι τοῖς ἕως θανάτου ἀποθνῄσκουσιν, ἵνα μὴ ἐξομόσωνται\footnoteA{ἐξόμνυμι, renounce by oath, 3rd pl. aor. mid. subj.} μηδὲ μέχρι φωνῆς τὸν χριστιανισμὸν\footnoteA{χριστιανισμός, Christianity, acc. sg. masc.}, καὶ παρεσκευασμένοις\footnoteA{παρασκευάζω, prepare, make ready, perf. mid. part. dat. pl. masc.} πρὸς πᾶσαν αἰκίαν\footnoteA{αἰκία, ill-treatment, torture, acc. sg. fem.} καὶ πάντα τρόπον θανάτου· ὡς δειλὸν γένος. λέγει δ' ἡμᾶς εἶναι καὶ φιλοσώματον\footnoteA{φιλοσώματος, body-loving, acc. sg. neut.} γένος, τοὺς φάσκοντας· «εἰ καὶ Χριστόν ποτε κατὰ σάρκα ἐγνώκαμεν, ἀλλὰ νῦν οὐκέτι γινώσκομεν» καὶ οὕτω προχείρως\footnoteA{προχείρως, readily, easily, adv.} ὑπὲρ εὐσεβείας τιθέντας τὸ σῶμα, ὡς οὐδὲ τὸ ἱμάτιον\footnoteA{ἱμάτιον, cloak, garment, acc. sg. neut.} ἀποδύσαιτ'\footnoteA{ἀποδύω, take off, strip off, 3rd sg. aor. mid. opt.} ἂν εὐχερῶς\footnoteA{εὐχερῶς, easily, readily, adv.} φιλόσοφος.