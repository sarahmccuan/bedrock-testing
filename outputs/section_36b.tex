\section*{7.1}

καὶ πρῶτόν γε λεκτέον πρὸς τὴν προσωποποιΐαν\footnoteA{προσωποποιΐα, personification, prosopopoeia, acc. sg. fem.} αὐτοῦ, περιτιθέντος\footnoteA{περιτίθημι, place around, attribute to, gen. sg. masc. pres. act. part.} ἡμῖν λόγους ὡς ὑφ' ἡμῶν λεγομένους εἰς τὴν περὶ ἀναστάσεως σαρκὸς ἀπολογίαν\footnoteA{ἀπολογία, defense, apology, acc. sg. fem.}, ὅτι ἀρετὴ μὲν προσωποποιοῦντός\footnoteA{προσωποποιέω, make prosopopoeia, employ personification, gen. sg. masc. pres. act. part.} ἐστι τηρῆσαι\footnoteA{τηρέω, keep, preserve, aor. act. inf.} τὸ βούλημα\footnoteA{βούλημα, will, intention, acc. sg. neut.} καὶ τὸ ἦθος τοῦ προσωποποιουμένου\footnoteA{προσωποποιέω, be personified, gen. sg. masc. pres. pass. part.}, κακία δὲ, ὅτε τὰ μὴ ἁρμόζοντά\footnoteA{ἁρμόζω, be fitting, be suitable, acc. pl. neut. pres. act. part.} τις περιτίθησι ῥήματα τῷ προσώπῳ τοῦ λέγοντος. καὶ ἐπ' ἴσης\footnoteA{ἐπ' ἴσης, equally, in the same way, adv.} γε ψεκτοὶ\footnoteA{ψεκτός, blameworthy, censurable, nom. pl. masc.} οἱ ἐν προσωποποιΐᾳ βαρβάροις καὶ ἀπαιδεύτοις\footnoteA{ἀπαίδευτος, uneducated, ignorant, dat. pl. masc.} ἢ οἰκότριψι\footnoteA{οἰκότριψ, house-worn slave, homegrown slave, dat. pl. masc.} καὶ μηδέ ποτε φιλοσόφων λόγων ἀκηκοόσι\footnoteA{ἀκούω, hear, listen to, dat. pl. masc. perf. act. part.} μηδὲ εὖ εἰρηκόσιν\footnoteA{εἶπον, speak well, dat. pl. masc. perf. act. part.} αὐτοὺς περιτιθέντες φιλοσοφίαν, ἣν ἔμαθε μὲν ὁ προσωποποιῶν, οὐκ εἰκὸς δὲ ἦν εἰδέναι τὸν προσωποποιούμενον, καὶ πάλιν αὖ οἱ τοῖς καθ' ὑπόθεσιν\footnoteA{κατὰ ὑπόθεσιν, hypothetically, by supposition, adv.} ὑποκειμένοις\footnoteA{ὑπόκειμαι, be subject to, be supposed, dat. pl. masc. pres. mid. part.} σοφοῖς καὶ τὰ θεῖα ἐγνωκόσι\footnoteA{γιγνώσκω, know, understand, dat. pl. masc. perf. act. part.} περιτιθέντες τὰ ἀπὸ ἰδιωτικῶν\footnoteA{ἰδιωτικός, private, common, ignorant, gen. pl. neut.} παθῶν ὑπὸ ἀπαιδεύτων λεγόμενα καὶ ἀπὸ ἀγνοίας ἀπαγγελλόμενα\footnoteA{ἀπαγγέλλω, report, announce, acc. pl. neut. pres. pass. part.}. ὅθεν Ὅμηρος μὲν ἐν πολλοῖς θαυμάζεται, τηρήσας τὰ τῶν ἡρώων πρόσωπα, ὁποῖα αὐτὰ ὑπέθετο\footnoteA{ὑποτίθημι, set down, establish, 3rd sg. aor. mid. ind.} ἀπ' ἀρχῆς, οἷον τὸ Νέστορος\footnoteA{Νέστωρ, Nestor, gen. sg. masc.} ἢ τὸ Ὀδυσσέως\footnoteA{Ὀδυσσεύς, Odysseus, gen. sg. masc.} ἢ τὸ Διομήδους\footnoteA{Διομήδης, Diomedes, gen. sg. masc.} ἢ τὸ Ἀγαμέμνονος\footnoteA{Ἀγαμέμνων, Agamemnon, gen. sg. masc.} ἢ Τηλεμάχου\footnoteA{Τηλέμαχος, Telemachus, gen. sg. masc.} ἢ Πηνελόπης\footnoteA{Πηνελόπη, Penelope, gen. sg. fem.} ἤ τινος τῶν λοιπῶν· Εὐριπίδης\footnoteA{Εὐριπίδης, Euripides, nom. sg. masc.} δὲ ὑπὸ Ἀριστοφάνους\footnoteA{Ἀριστοφάνης, Aristophanes, gen. sg. masc.} κωμῳδεῖται\footnoteA{κωμῳδέω, mock in comedy, ridicule, 3rd sg. pres. pass. ind.} ὡς ἀκαιροῤῥήμων\footnoteA{ἀκαιρόρρημος, speaking at wrong time, inopportune in speech, gen. sg. masc.} διὰ τὸ πολλάκις περιτεθεικέναι\footnoteA{περιτίθημι, place around, attribute to, perf. act. inf.} λόγους δογμάτων, ὧν ἀπὸ Ἀναξαγόρου\footnoteA{Ἀναξαγόρας, Anaxagoras, gen. sg. masc.} ἤ τινος ἔμαθε τῶν σοφῶν, βαρβάροις γυναιξὶν ἢ οἰκέταις\footnoteA{οἰκέτης, house slave, servant, dat. pl. masc.}.