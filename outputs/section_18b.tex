\section*{7.18}

\begin{greek}
\raggedright

εἴπερ δὲ ἀνέγνω\footnoteA{ἀναγιγνώσκω, read, 3rd sg. aor. act. ind.} τὸν Μωϋσέως\footnoteA{Μωϋσῆς, Moses, gen. sg. masc.} νόμον ὁ Κέλσος\footnoteA{Κέλσος, Celsus, nom. sg. masc.}, εἰκὸς ὅτι τὸ «δανιεῖς\footnoteA{δανίζω, lend, 2nd sg. fut. act. ind.} ἔθνεσι πολλοῖς, σὺ δὲ οὐ δανιῇ\footnoteA{δανίζομαι, borrow, 2nd sg. fut. mid. ind.},» λεγόμενον πρὸς τὸν τηροῦντα τὸν νόμον, ᾠήθη\footnoteA{οἴομαι, think, suppose, 3rd sg. aor. mid. ind.} τοιοῦτον εἶναι, ὥστ' ἐν ἐπαγγελίᾳ\footnoteA{ἐπαγγελία, promise, dat. sg. fem.} λέγεσθαι τῷ δικαίῳ τοσοῦτον πλουτήσειν\footnoteA{πλουτέω, be wealthy, aor. inf. act.} τὸν τυφλὸν πλοῦτον, ὥστε διὰ τὸ πλῆθος τῶν χρημάτων οὐ μόνον Ἰουδαίοις δανείζειν\footnoteA{δανίζω, lend, pres. inf. act.} τὸν δίκαιον ἀλλ' οὐδ' ἄλλῳ ἑνὶ ἔθνει ἢ δευτέρῳ ἢ τρίτῳ ἀλλὰ πολλοῖς. πόσα οὖν ἂν ὁ δίκαιος ἐκέκτητο\footnoteA{κτάομαι, possess, acquire, 3rd sg. plupf. mid. ind.} χρήματα, μισθὸν τῆς δικαιοσύνης αὐτὰ κατὰ τὸν νόμον λαβὼν, ἵνα δανιοῖ\footnoteA{δανίζω, lend, 3rd sg. pres. act. opt.} «πολλοῖς ἔθνεσιν»; ἀκόλουθον δ' ἐστὶ τῇ τοιαύτῃ ἐκδοχῇ\footnoteA{ἐκδοχή, interpretation, expectation, dat. sg. fem.} καὶ τὸ ὑπολαμβάνειν\footnoteA{ὑπολαμβάνω, suppose, assume, pres. inf. act.} ὅτι οὐδέ ποτε ὁ δίκαιος δανείζεται\footnoteA{δανίζομαι, borrow, 3rd sg. pres. mid. ind.}, ἐπεὶ γέγραπται· «σὺ δὲ οὐ δανιῇ.» ἆρ' οὖν ἔμεινε\footnoteA{μένω, remain, 3rd sg. aor. act. ind.} τὸ ἔθνος τοσούτοις χρόνοις ἐν τῇ κατὰ Μωϋσέα θεοσεβείᾳ\footnoteA{θεοσέβεια, piety, reverence for God, dat. sg. fem.}, προφανῶς βλέπον ψευδόμενον ὅσον ἐπὶ τῷ Κέλσῳ τὸν νομοθέτην\footnoteA{νομοθέτης, lawgiver, acc. sg. masc.}; οὐδὲ γὰρ ἱστόρηταί\footnoteA{ἱστορέω, record, relate, 3rd sg. perf. pass. ind.} τις τοσοῦτον πλουτήσας\footnoteA{πλουτέω, become wealthy, aor. act. part. nom. sg. masc.}, ὡς δεδανεικέναι\footnoteA{δανίζω, lend, perf. inf. act.} «ἔθνεσι πολλοῖς.» ἀλλ' οὐ πιθανὸν οὕτως αὐτοὺς διδασκομένους ἀκούειν τοῦ νόμου, ὡς Κέλσος ᾤετο\footnoteA{οἴομαι, think, suppose, 3rd sg. imperf. mid. ind.}, καὶ προφανῶς βλέποντας ψευδεῖς <τὰς> κατὰ τὸν νόμον ἐπαγγελίας ἀγωνίζεσθαι\footnoteA{ἀγωνίζομαι, struggle, contend, pres. inf. mid.} περὶ τοῦ νόμου.

ἐὰν δὲ τὰς ἀναγεγραμμένας\footnoteA{ἀναγράφω, record, write down, perf. pass. part. acc. pl. fem.} ἁμαρτίας τοῦ λαοῦ φέρῃ παράδειγμα τοῦ καταπεφρονηκέναι\footnoteA{καταφρονέω, despise, scorn, perf. inf. act.} αὐτοὺς τοῦ νόμου, τάχα διὰ τὸ κατεγνωκέναι\footnoteA{καταγιγνώσκω, condemn, judge against, perf. inf. act.} αὐτοὺς ὡς ψευδομένου, λεκτέον πρὸς αὐτὸν ὅτι ἀναγνωστέον\footnoteA{ἀναγιγνώσκω, read, verbal adj. neut. nom. sg.} καὶ τοὺς χρόνους, ἐν οἷς ὅλος ὁ λαὸς ἀναγέγραπται\footnoteA{ἀναγράφω, record, write down, 3rd sg. perf. pass. ind.} μετὰ τὸ πεποιηκέναι\footnoteA{ποιέω, do, make, perf. inf. act.} τὸ πονηρὸν ἐνώπιον κυρίου ἐπὶ τὸ βέλτιον καὶ τὴν κατὰ τὸν νόμον θεοσέβειαν μεταβεβληκέναι\footnoteA{μεταβάλλω, change, transform, perf. inf. act.}.

\end{greek}