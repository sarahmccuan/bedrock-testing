\section*{7.38}

Νοῦν τοίνυν ἢ ἐπέκεινα νοῦ καὶ οὐσίας λέγοντες εἶναι ἁπλοῦν καὶ ἀόρατον καὶ ἀσώματον τὸν τῶν ὅλων θεὸν, οὐκ ἂν ἄλλῳ τινὶ ἢ τῷ κατὰ τὴν ἐκείνου τοῦ νοῦ εἰκόνα γενομένῳ φήσομεν καταλαμβάνεσθαι τὸν θεόν· νῦν μὲν, ἵνα τῇ λέξει χρήσωμαι τοῦ Παύλου, «δι' ἐσόπτρου\footnoteA{ἔσοπτρον, mirror, gen. sg. neut.} καὶ <ἐν> αἰνίγματι\footnoteA{αἴνιγμα, riddle, enigma, dat. sg. neut.} τότε δὲ πρόσωπον πρὸς πρόσωπον.» «πρόσωπον» δὲ ἐὰν λέγω, μὴ συκοφαντείτω\footnoteA{συκοφαντέω, slander, falsely accuse, 3rd sg. pres. act. impv.} τις διὰ τὴν λέξιν τὸν δηλούμενον νοῦν ὑπ' αὐτῆς, ἀλλὰ μανθανέτω <ἐν τῷ> «ἀνακεκαλυμμένῳ\footnoteA{ἀνακαλύπτω, unveil, uncover, perf. mid. part. dat. sg. neut.} προσώπῳ τὴν δόξαν κυρίου κατοπτριζόμενοι\footnoteA{κατοπτρίζομαι, reflect as in a mirror, contemplate, 1st pl. pres. mid. part. nom.} καὶ τὴν αὐτὴν εἰκόνα μεταμορφούμενοι\footnoteA{μεταμορφόω, transform, change form, 1st pl. pres. pass. part. nom.} ἀπὸ δόξης εἰς δόξαν» οὐ πρόσωπον αἰσθητὸν ἐν τοῖς τοιούτοις παραλαμβανόμενον ἀλλὰ κατὰ τροπολογίαν\footnoteA{τροπολογία, figurative speech, metaphor, acc. sg. fem.} νοούμενον ὡς καὶ ὀφθαλμοὺς καὶ ὦτα. καὶ ὅσα ἐν τοῖς ἀνωτέρω ὁμώνυμα\footnoteA{ὁμώνυμος, having the same name, homonymous, acc. pl. neut.} τοῖς τοῦ σώματος μέλεσι παρεστήσαμεν\footnoteA{παρίστημι, present, set before, 1st pl. aor. act. ind.}.

καὶ ἄνθρωπος μὲν οὖν, τουτέστι ψυχὴ χρωμένη σώματι, λεγομένη «ὁ ἔσω ἄνθρωπος» ἀλλὰ καὶ «ψυχὴ,» ἀποκρίνεται οὐχ ἅπερ Κέλσος ἀνέγραψεν\footnoteA{ἀναγράφω, write down, record, 3rd sg. aor. act. ind.}, ἀλλ' ἅπερ αὐτὸς διδάσκει ὁ τοῦ θεοῦ ἄνθρωπος. σαρκὸς δὲ φωνῇ οὐκ ἂν Χριστιανὸς χρήσαιτο, μαθὼν «πνεύματι τὰς πράξεις τοῦ σώματος» θανατοῦν\footnoteA{θανατόω, put to death, make dead, pres. act. inf.} καὶ «πάντοτε τὴν νέκρωσιν\footnoteA{νέκρωσις, death, deadness, acc. sg. fem.} τοῦ Ἰησοῦ ἐν τῷ σώματι» περιφέρειν\footnoteA{περιφέρω, carry about, bear around, pres. act. inf.} καὶ «νεκρώσατε\footnoteA{νεκρόω, make dead, put to death, 2nd pl. aor. act. impv.} τὰ μέλη τὰ ἐπὶ τῆς γῆς,» καὶ εἰδὼς τί δηλοῦται ἐκ τοῦ «οὐ μὴ καταμείνῃ\footnoteA{καταμένω, remain, continue, 3rd sg. aor. act. subj.} τὸ πνεῦμά μου ἐν τοῖς ἀνθρώποις τούτοις εἰς τὸν αἰῶνα διὰ τὸ εἶναι αὐτοὺς σάρκας,» ἐπιστάμενος\footnoteA{ἐπίσταμαι, know, understand, pres. mid. part. nom. sg. masc.} δὲ καὶ ὅτι «οἱ ἐν σαρκὶ ὄντες θεῷ ἀρέσαι\footnoteA{ἀρέσκω, please, be pleasing to, aor. act. inf.} οὐ δύνανται» καὶ διὰ τοῦτο πάντα πράττων εἰς τὸ μηδαμῶς\footnoteA{μηδαμῶς, by no means, in no way, adv.} αὐτὸν ἔτι εἶναι «ἐν τῇ σαρκὶ ἀλλ'» «ἐν» μόνῳ «τῷ πνεύματι.»