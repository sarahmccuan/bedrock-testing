\section*{XVI.}

\begin{greek}
\raggedright

Ἀλλ' οὐδ' ἅπερ ἐν ὑποθέσει παρειλήφαμεν\footnoteA{παραλαμβάνω, receive, accept, 1st pl. perf. act. ind.} παραπλήσιά\footnoteA{παραπλήσιος, similar, comparable, nom. pl. neut.} ἐστι ταῖς περὶ Ἰησοῦ προφητείαις. οὐ γὰρ προεῖπον\footnoteA{προλέγω, foretell, predict, 3rd pl. aor. act. ind.} αἱ προφητεῖαι θεὸν σταυρωθήσεσθαι\footnoteA{σταυρόω, crucify, fut. pass. inf.}, αἵτινές φασι περὶ τοῦ ἀναδεξαμένου\footnoteA{ἀναδέχομαι, undertake, accept, aor. mid. part. gen. sg. masc.} τὸν θάνατον· «καὶ εἴδομεν αὐτὸν, καὶ οὐκ εἶχεν εἶδος οὐδὲ κάλλος\footnoteA{κάλλος, beauty, acc. sg. neut.}, ἀλλὰ τὸ εἶδος αὐτοῦ ἄτιμον\footnoteA{ἄτιμος, dishonored, without honor, nom. sg. neut.}, ἐκλεῖπον\footnoteA{ἐκλείπω, fail, be lacking, pres. act. part. nom. sg. neut.} παρὰ τοὺς υἱοὺς τῶν ἀνθρώπων· ἄνθρωπος ἐν πληγῇ\footnoteA{πληγή, blow, affliction, dat. sg. fem.} ὢν καὶ πόνῳ\footnoteA{πόνος, toil, suffering, dat. sg. masc.} καὶ εἰδὼς φέρειν μαλακίαν\footnoteA{μαλακία, sickness, weakness, acc. sg. fem.}.» ὅρα οὖν <ὡς> σαφῶς\footnoteA{σαφῶς, clearly, plainly, adv.} ἄνθρωπον εἰρήκασι τὸν πεπονθότα\footnoteA{πάσχω, suffer, perf. act. part. acc. sg. masc.} ἀνθρώπινα\footnoteA{ἀνθρώπινος, human, acc. pl. neut.}. καὶ αὐτὸς ἀκριβῶς\footnoteA{ἀκριβῶς, accurately, precisely, adv.} εἰδὼς Ἰησοῦς ὅτι τὸ ἀποθνῇσκον\footnoteA{ἀποθνῄσκω, die, pres. act. part. nom. sg. neut.} ἄνθρωπος ἦν, φησὶ πρὸς τοὺς ἐπιβουλεύοντας\footnoteA{ἐπιβουλεύω, plot against, pres. act. part. acc. pl. masc.} αὐτῷ· «νῦν δὲ ζητεῖτέ με ἀποκτεῖναι, ἄνθρωπον, ὃς τὴν ἀλήθειαν ὑμῖν λελάληκα\footnoteA{λαλέω, speak, tell, 1st sg. perf. act. ind.}, ἣν ἤκουσα ἀπὸ τοῦ θεοῦ.» εἰ δέ τι θεῖον\footnoteA{θεῖος, divine, nom. sg. neut.} ἐν τῷ κατ' αὐτὸν νοουμένῳ\footnoteA{νοέω, conceive, understand, pres. pass. part. dat. sg. masc.} ἀνθρώπῳ ἐτύγχανεν\footnoteA{τυγχάνω, happen to be, 3rd sg. imperf. act. ind.}, ὅπερ ἦν ὁ μονογενὴς\footnoteA{μονογενής, only-begotten, nom. sg. masc.} τοῦ θεοῦ καὶ ὁ «πρωτότοκος\footnoteA{πρωτότοκος, firstborn, nom. sg. masc.} πάσης κτίσεως\footnoteA{κτίσις, creation, gen. sg. fem.},» ὁ λέγων· «ἐγώ εἰμι ἡ ἀλήθεια» καὶ «ἐγώ εἰμι ἡ ζωὴ» καὶ «ἐγώ εἰμι ἡ θύρα» καὶ «»ἐγώ εἰμι ἡ ὁδὸς καὶ «ἐγώ εἰμι ὁ ἄρτος ὁ ζῶν ὁ ἐκ τοῦ οὐρανοῦ καταβάς\footnoteA{καταβαίνω, come down, descend, aor. act. part. nom. sg. masc.}»· ἄλλος δή που ὁ περὶ τούτου καὶ τῆς οὐσίας\footnoteA{οὐσία, essence, substance, gen. sg. fem.} αὐτοῦ λόγος ἐστὶ παρὰ τὸν περὶ τοῦ νοουμένου\footnoteA{νοέω, conceive, understand, pres. pass. part. gen. sg. masc.} κατὰ τὸν Ἰησοῦν ἀνθρώπου.

διόπερ\footnoteA{διόπερ, wherefore, for which reason, conj.} οὐδ' οἱ πάνυ\footnoteA{πάνυ, very, exceedingly, adv.} ἁπλούστατοι\footnoteA{ἁπλοῦς, simple, superlative nom. pl. masc.} καὶ λόγοις οὐκ ἐντεθραμμένοι\footnoteA{ἐντρέφω, nourish in, train in, perf. pass. part. nom. pl. masc.} ἐξεταστικοῖς\footnoteA{ἐξεταστικός, investigative, critical, dat. pl. masc.} Χριστιανοὶ εἴποιεν ἂν τεθνηκέναι\footnoteA{θνῄσκω, die, perf. act. inf.} τὴν ἀλήθειαν ἢ τὴν ζωὴν ἢ τὴν ὁδὸν ἢ τὸν ἐξ οὐρανοῦ καταβάντα\footnoteA{καταβαίνω, come down, descend, aor. act. part. acc. sg. masc.} ζῶντα ἄρτον ἢ τὴν ἀνάστασιν· φησὶ γὰρ ἑαυτὸν ἀνάστασιν εἶναι ὁ ἐν τῷ φαινομένῳ\footnoteA{φαίνω, appear, pres. pass. part. dat. sg. masc.} ἀνθρώπῳ κατὰ τὸν Ἰησοῦν διδάξας\footnoteA{διδάσκω, teach, aor. act. part. nom. sg. masc.} τὸ «ἐγώ εἰμι ἡ ἀνάστασις.» ἀλλὰ καὶ οὐδεὶς <οὕτως> ἐμβρόντητος\footnoteA{ἐμβρόντητος, thunderstruck, foolish, nom. sg. masc.} ἡμῶν ἐστιν, ἵν' εἴπῃ· τέθνηκεν\footnoteA{θνῄσκω, die, 3rd sg. perf. act. ind.} «ἡ ζωὴ» ἢ· «τέθνηκεν\footnoteA{θνῄσκω, die, 3rd sg. perf. act. ind.} ἡ ἀνάστασιν.» ἦν δ' ἂν τὸ τῆς ὑποθέσεως τοῦ Κέλσου χώραν\footnoteA{χώρα, place, room, acc. sg. fem.} ἔχον, εἰ ἐφάσκομεν\footnoteA{φάσκω, claim, assert, 1st pl. imperf. act. ind.} προειρηκέναι\footnoteA{προλέγω, foretell, predict, perf. act. inf.} τοὺς προφήτας τεθνήξεσθαι\footnoteA{θνῄσκω, die, fut. act. inf.} τὸν θεὸν λόγον ἢ τὴν ἀλήθειαν ἢ τὴν ζωὴν ἢ τὴν ἀνάστασιν ἤ τι τῶν ἄλλων, ἅ φησιν εἶναι ὁ υἱὸς τοῦ θεοῦ.

\end{greek}