\section*{XXIII.}

Σαφῆ μὲν οὖν καὶ ἐκ τούτων, πῶς Ἰησοῦς «ὁ Ναζωραῖος»\footnoteA{Ναζωραῖος, Nazarene, nom. sg. masc.} ἄνθρωπος οὐκ ἀντινομοθετεῖ\footnoteA{ἀντινομοθετέω, legislate against, contradict, 3rd sg. pres. act. ind.} τοῖς περὶ πλούτου καὶ τῶν ἐξισταμένων\footnoteA{ἐξίστημι, be amazed, be astonished, pres. mid./pass. part. gen. pl.} εἰρημένοις περὶ τοῦ δύσκολον\footnoteA{δύσκολος, difficult, hard, acc. sg. neut.} εἶναι πλούσιον εἰσέρχεσθαι εἰς τὴν τοῦ θεοῦ βασιλείαν· εἴτε πλούσιόν τις λαμβάνει ἁπλούστερον\footnoteA{ἁπλοῦς, simple, plain, comp. acc. sg. masc.} τὸν ὑπὸ πλούτου περισπώμενον\footnoteA{περισπάω, distract, draw away, pres. pass. part. acc. sg. masc.} καὶ ὡς ὑπὸ ἀκάνθης\footnoteA{ἄκανθα, thorn, gen. sg. fem.} αὐτοῦ ἐμποδιζόμενον\footnoteA{ἐμποδίζω, hinder, impede, pres. pass. part. acc. sg. masc.} φέρειν τοὺς τοῦ λόγου καρποὺς εἴτε καὶ τὸν ἐν τοῖς ψευδέσι δόγμασι\footnoteA{δόγμα, doctrine, opinion, dat. pl. neut.} πλουτοῦντα\footnoteA{πλουτέω, be rich, be wealthy, pres. act. part. acc. sg. masc.}, περὶ οὗ ἐν Παροιμίαις\footnoteA{παροιμία, proverb, dat. pl. fem.} γέγραπται· «κρείσσων\footnoteA{κρείσσων, better, stronger, nom. sg. masc./fem.} πτωχὸς δίκαιος ἢ πλούσιος ψεύστης\footnoteA{ψεύστης, liar, nom. sg. masc.}.»

εἰκὸς δὲ ἀπὸ τοῦ «ὁ θέλων ἐν ὑμῖν εἶναι πρῶτος ἔστω πάντων διάκονος\footnoteA{διάκονος, servant, minister, nom. sg. masc.}» καὶ «οἱ ἄρχοντες τῶν ἐθνῶν κατακυριεύουσιν\footnoteA{κατακυριεύω, lord it over, dominate, 3rd pl. pres. act. ind.} αὐτῶν» καὶ «οἱ ἐξουσιάζοντες\footnoteA{ἐξουσιάζω, exercise authority, 3rd pl. pres. act. part. nom.} ἐν αὐτοῖς εὐεργέται\footnoteA{εὐεργέτης, benefactor, nom. pl. masc.} καλοῦνται» εἰληφέναι\footnoteA{λαμβάνω, take, receive, perf. act. inf.} τὸν Κέλσον ὅτι Ἰησοῦς φιλαρχίαν\footnoteA{φιλαρχία, love of rule, ambition, acc. sg. fem.} ἐκώλυεν\footnoteA{κωλύω, hinder, prevent, 3rd sg. impf. act. ind.}, ἥντινα οὐκ ἐναντίαν εἶναι νομιστέον\footnoteA{νομίζω, think, consider, verbal adj. acc. sg. fem.} τοῦ «ἄρξεις σὺ ἐθνῶν πολλῶν, σοῦ δὲ οὐκ ἄρξουσι,» μάλιστα διὰ τὰ ἀποδεδομένα\footnoteA{ἀποδίδωμι, give back, render, perf. pass. part. acc. pl. neut.} εἰς τὴν λέξιν.

ἑξῆς δὲ τούτοις παραῤῥίπτει\footnoteA{παραρρίπτω, throw in casually, remark, 3rd sg. pres. act. ind.} περὶ τῆς σοφίας ὁ Κέλσος, οἰόμενος\footnoteA{οἴομαι, think, suppose, pres. mid. part. nom. sg. masc.} τὸν Ἰησοῦν διδάσκειν μὴ παριτητὸν\footnoteA{παρίτητος, accessible, approachable, acc. sg. masc.} εἶναι πρὸς τὸν πατέρα τῷ σοφῷ. εἴπωμεν δὲ πρὸς αὐτόν· ποίῳ σοφῷ; εἰ μὲν γὰρ τούτῳ, ὃς πεποίωται\footnoteA{ποιέω, make, create, 3rd sg. perf. mid./pass. ind.} κατὰ τὴν λεγομένην σοφίαν «τοῦ κόσμου τούτου,» οὖσαν μωρίαν\footnoteA{μωρία, foolishness, acc. sg. fem.} «παρὰ τῷ θεῷ,» καὶ ἡμεῖς φήσομεν μὴ παριτητὸν εἶναι πρὸς τὸν πατέρα τῷ οὕτως σοφῷ· εἰ δὲ σοφίαν τις νοήσαι\footnoteA{νοέω, understand, think, aor. act. inf.} τὸν Χριστὸν, ἐπεὶ Χριστός ἐστι θεοῦ δύναμις καὶ θεοῦ σοφία, οὐ μόνον παριτητὸν πρὸς τὸν πατέρα τῷ οὕτως σοφῷ λέγομεν εἶναι, ἀλλὰ καὶ πολλῷ τῶν μὴ τοιούτων διαφέρειν τὸν κεκοσμημένον\footnoteA{κοσμέω, adorn, equip, perf. pass. part. acc. sg. masc.} χαρίσματί\footnoteA{χάρισμα, gift, grace, dat. sg. neut.} φαμεν λόγῳ «σοφίας» καλουμένῳ, διὰ τοῦ πνεύματος διδομένῳ.